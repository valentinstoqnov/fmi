\documentclass[10pt]{article}
\usepackage[utf8]{inputenc} 
\usepackage{amsmath}
\usepackage[utf8]{inputenc}
\usepackage[bulgarian]{babel}
\usepackage{amssymb}

\newcommand*{\Z}{\mathbb{Z}}
\newcommand*{\Q}{\mathbb{Q}}
\newcommand{\triq}{\; \underline{\triangleleft} \;}

\begin{document}

\title{Решение на домашно 2}
\author{Валентин Стоянов}
\date{март 2018}
\maketitle

\section*{Задача 1.}
Нека $R = \{a, b, c, d\}$ е пръстен с таблици за събиране и умножение, съответно,\\

\begin{tabular}{ |c|c|c|c|c| } 
	\hline
	+ & a & b & c & d \\ 
	\hline
	a & a & b & c & d \\
	\hline
	b & b & a & d & c \\
	\hline
	c & c & d & a & b \\
	\hline
	d & d & c & b & a\\
	\hline
\end{tabular}
\quad
и
\quad
\begin{tabular}{ |c|c|c|c|c| } 
	\hline
	$\ast$ & a & b & c & d \\ 
	\hline
	a & a & a & a & a \\
	\hline
	b & a & $\bullet$ & a & b \\
	\hline
	c & a & $\bullet$ & $\bullet$ & c \\
	\hline
	d & a & d & $\bullet$ & $\bullet$ \\
	\hline
\end{tabular}
\\
\\
\\
Да се определят отбелязаните с кръгче елементи в таблицата за умножение и да се намерят идеалите на пръстена $R$.
\subsection*{Решение:}
$dd = (b + c)d = bd + cd = b + c = d$\\
$dc = d(d + b) = dd + db = d + d = a$\\
$bb = b(c + d) = bc + bd = a + b = b$\\
$cb = (b + d)b = bb + db = b + d = c$\\
$cc = c(b + d) = cb + cd = c + c = a$\\
След попълване, таблицата за умножение изглежда така: \quad
\begin{tabular}{ |c|c|c|c|c| } 
	\hline
	$\ast$ & a & b & c & d \\ 
	\hline
	a & a & a & a & a \\
	\hline
	b & a & b & a & b \\
	\hline
	c & a & c & a & c \\
	\hline
	d & a & d & a & d \\
	\hline
\end{tabular}
\\
Идеалите в пръстена са: $\{a\}$, $\{a, b, c, d\}$ и $\{a, c\}$.\\
\newpage

\section*{Задача 2.}
Разглеждаме множествата
\begin{center}
	$A = \{\frac{f}{g} \mid f, g \in \Q[x] \,;\, g(45466) \neq 0\}$\quad и\quad$M = \{\frac{f}{g} \in A \mid f(45466) = 0\}$.
\end{center}
Да се докаже, че $A$ е пръстен (относно обичайните операции: събиране и умножение на рационални
функции), $M$ е идеал на $A$, който съдържа всеки собствен идеал на $A$ и $A/M \cong \Q$
\subsection*{Решение:}
Нека $\frac{f_1}{g_1}, \frac{f_2}{g_2} \in A$.
Ще проверим дали $\frac{f_1}{g_1} - \frac{f_2}{g_2} \in A$ и $\frac{f_1}{g_1}\frac{f_2}{g_2} \in A$.\\
\begin{itemize}
	\item $\frac{f_1}{g_1} - \frac{f_2}{g_2} = \frac{f_1g_2 - f_2g_1}{g_1g_2}$, тъй като $\Q[x]$ е пръстен и $f_1, f_2, g_1, g_2 \in \Q[x]$, то следва, че $g_1g_2 \in \Q[x]$ и $f_1g_2 - f_2g_1 \in \Q[x]$. Следователно $\frac{f_1g_2 - f_2g_1}{g_1g_2} \in A$.
	\item Тъй като $\Q[x]$ е пръстен следва, че $f_1f_2 \in \Q[x]$ и $g_1g_2 \in \Q[x]$. Следователно $\frac{f_1f_2}{g_1g_2} \in A$
\end{itemize}
$\Rightarrow A$ е пръстен.\\
\\
$M$ съдържа необратимите елементи на $A$\\
$A\textbackslash M$ съдържа обратимите на $A$.\\
$I$ е идеал. Ако съществува $a \in I$, който е обратим, то $I = A$. Следователно всеки собствен идеал на $A$ се съдържа в $M$.\\
\\
Нека $\varphi: A \to \Q$ такова, че $\varphi(\frac{f}{g}) = (\frac{f}{g})(45466)$.\\
Ще проверим дали $\varphi$ е хомоморфизъм на пръстени. Нека $\frac{f_1}{g_1}, \frac{f_2}{g_2} \in A$.
\begin{itemize}
	\item $\varphi(\frac{f_1}{g_1} + \frac{f_2}{g_2}) = \varphi(\frac{f_1g_2 + f_2g_1}{g_1g_2}) = (\frac{f_1g_2 + f_2g_1}{g_1g_2})(45466) = (\frac{f_1}{g_1})(45466) + (\frac{f_2}{g_2})(45466) = \\ = \varphi(\frac{f_1}{g_1}) + \varphi(\frac{f_2}{g_2})$
	\item $\varphi(\frac{f_1}{g_1}\frac{f_2}{g_2}) = \varphi(\frac{f_1f_2}{g_1g_2}) = (\frac{f_1f_2}{g_1g_2})(45466) = (\frac{f_1}{g_1})(45466)(\frac{f_2}{g_2})(45466) = \varphi(\frac{f_1}{g_1})\varphi(\frac{f_2}{g_2})$
\end{itemize}
$\Rightarrow \varphi$ е хомоморфизъм.\\
От начина, по който е зададено $М$ следва, че $Ker\varphi = M$.\\
Съгласно Теоремата за хомоморфизми на пръстени: $A/M \cong \Q$
\newpage

\section*{Задача 4.}
Да се докаже, че факторпръстенът $\Z_3[x]/(x^3 + \bar{2}x + \bar{1})$ е поле. Намерете обратния елемент на $\bar{2}x^2 + x + \bar{1}$.
\subsection*{Решение:}
Нека $f = x^3 + \bar{2}x + \bar{1}$.\\
$\Z_3[x]/(f)$ е поле, ако $f$ е неразложим над $\Z_3$.\\
Понеже $deg(f) = 3$, то $f$ е неразложим $\Leftrightarrow f$ няма корен в $\Z_3$.\\
$\Z_3 = \{\bar{0}, \bar{1}, \bar{2}\}$;
$\quad f(\bar{0}) = \bar{1} \neq 0 \quad f(\bar{1}) = \bar{1} \neq 0 \quad f(\bar{2}) = \bar{1} \neq 0 \Rightarrow$\\
$\Rightarrow f$ няма корени в $\Z_3 \Rightarrow f$ е неразложим над $\Z_3 \Rightarrow \Z_3[x]/(f)$ е поле.\\\\
Нека $g = \bar{2}x^2 + x + \bar{1}$. Искаме да намерим обратния елемент на $g$ в $\Z_3[x]/(f)$, т.е търсим такъв полином $h$, че $gh \equiv 1 \pmod{f}$.\\
$gh \equiv 1 \pmod{f} \Leftrightarrow gh + kf = 1$, за някое $k \in \Z_3[x]$.\\
$f = g(\bar{2}x + \bar{2}) + (x + \bar{2})\\
g = (x + \bar{2})\bar{2}x + \bar{1}\\
x + \bar{2} = \bar{1}(x + \bar{2}) + \bar{0}\\
\Rightarrow g(x^2 + x + \bar{1}) + fx = 1\\
\Rightarrow h = x^2 + x + \bar{1}$
\newpage

\section*{Задача 5.}
Нека $K$ е комутативен пръстен с единица. Да се докаже, че\\
а) Ако $I \triq K$, то $M_n(I) \triq M_n(K)$ и $M_n(K)/M_n(I) \cong M_n(K/I)$;\\
б) Всеки идеал $J \triq M_n(K)$ е от вида $J = M_n(I)$, където $I \triq K$.
\subsection*{Решение:}
а)
\begin{itemize}
	\item Нека $(a_{ij})_{n \times n}, (b_{ij})_{n \times n} \in M_n(I) \quad ?\Rightarrow \quad (a_{ij})_{n \times n} - (b_{ij})_{n \times n} \in M_n(I)$\\$(a_{ij})_{n \times n} - (b_{ij})_{n \times n} = (a_{ij} - b_{ij})_{n \times n}$, $I$ е идеал $\Rightarrow a_{ij} - b_{ij} \in I \Rightarrow (a_{ij})_{n \times n} - (b_{ij})_{n \times n} \in M_n(I)$
	
	\item Нека $(a_{ij})_{n \times n} \in M_n(I)$ и $(k_{ij})_{n \times n} \in M_n(K) \quad ?\Rightarrow \quad (a_{ij})_{n \times n}(k_{ij})_{n \times n} \in M_n(I)$\\$(a_{ij})_{n \times n}(k_{ij})_{n \times n} = (a_{ij}k_{ij})_{n \times n}$, $I$ е идеал $\Rightarrow a_{ij}k_{ij} \in I \Rightarrow (a_{ij})_{n \times n}(k_{ij})_{n \times n} \in M_n(I)$
\end{itemize}
$\Rightarrow M_n(I) \triq M_n(K)$\\\\

С помощта на хомоморфизма $\psi: K \to K/I$ такъв, че $\psi(a) = \bar{a} = a + I$, дефинираме изображение $\varphi: M_n(K) \to M_n(K/I)$, действащо по правилото $\varphi((a_{ij})_{n \times n}) = (\bar{a}_{ij})_{n \times n}$. От това, че $\psi$ е хомоморфизъм на пръстени следва, че $\varphi$ е хомоморфизъм. Ясно е, че $Ker\varphi = M_n(I)$. От теоремата за хомоморфизмите на пръстени следва, че $M_n(K)/M_n(I) \cong M_n(K/I)$.\\\\
б) Нека $J \triq M_n(K)$ и $I$ да е подмножество на $K$, състоящо се от всички елементи на всички матрици на $J$. Ще докажем, че $I \triq K$ и $J = M_n(I)$.\\
Нека $a, b \in I$. Следователно съществуват матрици $A, B \in J$ такива, че $a$ и $b$ са съответно техни елементи. Нека $a$ е на $(i, j)$-то място в $A$ и $b$ е на $(p, q)$-то място в $B$. Нека $e_{ij}$ е матрицата с, която има единица на $(i, j)$-то място. Понеже $J \triq M_n(K)$, то $e_{1i}Ae_{j1}, e_{1p}Be_{q1} \in J$, откъдето следва, че и $e_{1i}Ae_{j1} - e_{1p}Be_{q1} \in J$. $e_{1i}Ae_{j1} = ae_{11}$, $e_{1p}Be_{q1} = be_{11} \Rightarrow ae_{11} - be_{11} = (a - b)e_{11} \in J \Rightarrow a - b \in I$. Нека $k \in K$. Тогава $(kE)A \in J$ и на $(i, j)$-то място в $(kE)A$ стои $ka$, следователно $ka = ak \in I$. Следователно $I \triq K$.

Нека $(c_{ij})_{n \times n} \in M_n(I)$. За всеки 2 индекса $(i, j)$ съществува матрица $A \in J$, за която $c_{ij}$ е елемент на $A$. Ако $c_{ij}$ стои на $(s, t)$-то място в $A$, то получаваме, че $c_{ij}e_{ij} = e_{is}Ae_{tj} \in J$. Тогава $(c_{ij})_{n \times n} = \sum_{i, j=1}^{n} c_{ij}e_{ij} \in J$. Следователно $M_n(I) \subseteq J$. Лесно се проверява и че $J \subseteq M_n(I)$. Откъдето следва, че $J = M_n(I)$.




\end{document}