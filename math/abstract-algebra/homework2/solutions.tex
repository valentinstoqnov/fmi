\documentclass[10pt]{article}
\usepackage[utf8]{inputenc} 
\usepackage{amsmath}
\usepackage[utf8]{inputenc}
\usepackage[bulgarian]{babel}
\usepackage{amssymb}

\newcommand*{\Q}{\mathbb{Q}}

\begin{document}

\title{Решение на домашно 2}
\author{Валентин Стоянов}
\date{март 2018}
\maketitle

\section*{Задача 1.}
Нека $R = \{a, b, c, d\}$ е пръстен с таблици за събиране и умножение, съответно,\\

\begin{tabular}{ |c|c|c|c|c| } 
	\hline
	+ & a & b & c & d \\ 
	\hline
	a & a & b & c & d \\
	\hline
	b & b & a & d & c \\
	\hline
	c & c & d & a & b \\
	\hline
	d & d & c & b & a\\
	\hline
\end{tabular}
\quad
и
\quad
\begin{tabular}{ |c|c|c|c|c| } 
	\hline
	$\ast$ & a & b & c & d \\ 
	\hline
	a & a & a & a & a \\
	\hline
	b & a & $\bullet$ & a & b \\
	\hline
	c & a & $\bullet$ & $\bullet$ & c \\
	\hline
	d & a & d & $\bullet$ & $\bullet$ \\
	\hline
\end{tabular}
\\
\\
\\
Да се определят отбелязаните с кръгче елементи в таблицата за умножение и да се намерят идеалите на пръстена $R$.
\subsection*{Решение:}
$dd = (b + c)d = bd + cd = b + c = d$\\
$dc = d(d + b) = dd + db = d + d = a$\\
$bb = b(c + d) = bc + bd = a + b = b$\\
$cb = (b + d)b = bb + db = b + d = c$\\
$cc = c(b + d) = cb + cd = c + c = a$\\
След попълване, таблицата за умножение изглежда така: \quad
\begin{tabular}{ |c|c|c|c|c| } 
	\hline
	$\ast$ & a & b & c & d \\ 
	\hline
	a & a & a & a & a \\
	\hline
	b & a & b & a & b \\
	\hline
	c & a & c & a & c \\
	\hline
	d & a & d & a & d \\
	\hline
\end{tabular}
\\
Идеалите в пръстена са: $\{a\}$, $\{a, b, c, d\}$ и $\{a, c\}$.\\
\newpage

\section*{Задача 2.}
Разглеждаме множествата
\begin{center}
	$A = \{\frac{f}{g} \mid f, g \in \Q[x] \,;\, g(45466) \neq 0\}$\quad и\quad$M = \{\frac{f}{g} \in A \mid f(45466) = 0\}$.
\end{center}
Да се докаже, че $A$ е пръстен (относно обичайните операции: събиране и умножение на рационални
функции), $M$ е идеал на $A$, който съдържа всеки собствен идеал на $A$ и $A/M \cong \Q$
\subsection*{Решение:}
Нека $\frac{f_1}{g_1}, \frac{f_2}{g_2} \in A$.
Ще проверим дали $\frac{f_1}{g_1} - \frac{f_2}{g_2} \in A$ и $\frac{f_1}{g_1}\frac{f_2}{g_2} \in A$.\\
\begin{itemize}
	\item $\frac{f_1}{g_1} - \frac{f_2}{g_2} = \frac{f_1g_2 - f_2g_1}{g_1g_2}$, тъй като $\Q[x]$ е пръстен и $f_1, f_2, g_1, g_2 \in \Q[x]$, то следва, че $g_1g_2 \in \Q[x]$ и $f_1g_2 - f_2g_1 \in \Q[x]$. Следователно $\frac{f_1g_2 - f_2g_1}{g_1g_2} \in A$.
	\item Тъй като $\Q[x]$ е пръстен следва, че $f_1f_2 \in \Q[x]$ и $g_1g_2 \in \Q[x]$. Следователно $\frac{f_1f_2}{g_1g_2} \in A$
\end{itemize}
$\Rightarrow A$ е пръстен.\\
\\
Нека $\varphi: A \to \Q$ такова, че $\varphi(\frac{f}{g}) = (\frac{f}{g})(x)$.\\
Ще проверим дали $\varphi$ е хомоморфизъм на пръстени. Нека $\frac{f_1}{g_1}, \frac{f_2}{g_2} \in A$.
\begin{itemize}
	\item $\varphi(\frac{f_1}{g_1} + \frac{f_2}{g_2}) = \varphi(\frac{f_1g_2 + f_2g_1}{g_1g_2}) = (\frac{f_1g_2 + f_2g_1}{g_1g_2})(x) = (\frac{f_1}{g_1})(x) + (\frac{f_2}{g_2})(x) = \\ = \varphi(\frac{f_1}{g_1}) + \varphi(\frac{f_2}{g_2})$
	\item $\varphi(\frac{f_1}{g_1}\frac{f_2}{g_2}) = \varphi(\frac{f_1f_2}{g_1g_2}) = (\frac{f_1f_2}{g_1g_2})(x) = (\frac{f_1}{g_1})(x)(\frac{f_2}{g_2})(x) = \varphi(\frac{f_1}{g_1})\varphi(\frac{f_2}{g_2})$
\end{itemize}
$\Rightarrow \varphi$ е хомоморфизъм.\\
От начина, по който е зададено $М$ следва, че $Ker\varphi = M$.\\
Съгласно Теоремата за хомоморфизми на пръстени: $A/M \cong \Q$

\end{document}