\documentclass[10pt]{article}
\usepackage[utf8]{inputenc} 
\usepackage{amsfonts}
\usepackage{amsmath}
\usepackage{amssymb}
\usepackage[utf8]{inputenc}
\usepackage[bulgarian]{babel}
\usepackage{mathtools}

\newcommand*{\Z}{\mathbb{Z}}
\newcommand*{\N}{\mathbb{N}}
\newcommand*{\R}{\mathbb{R}}
\newcommand*{\Q}{\mathbb{Q}}
\newcommand{\triq}{\; \underline{\triangleleft} \;}

\begin{document}

\title{Теория 2}
\author{Валентин Стоянов}
\date{май 2018}
\maketitle

\section*{Задача 1.}

\subsection*{Какво представлява класът остатъци $\bar{m} \in \Z_n$}
Това са числата, които при деление на $n$ дават остатък $m$.

\subsection*{За какви числа $n$ пръстенът $\Z_n$ е поле}
Пръстенът $\Z_n$ е поле, когато $n$ е просто число.

\subsection*{За какви числа $k$ класът остатъци $\bar{k} \in \Z_n$ е обратим елемент на пръстена $\Z_n$}
$(k, n) = 1$ 

\subsection*{За какви числа $k$ класът остатъци $\bar{k} \in \Z_n$ е делител на нулата в пръстена $\Z_n$}
$(k, n) \neq 1$ 

\subsection*{За какви числа $k$ класът остатъци $k \in \Z_n$ не е обратим елемент на пръстена $\Z_n$}
$(k, n) \neq 1$ 

\subsection*{За какви числа $k$ класът остатъци $k \in \Z_n$ не е делител на нулата в пръстена $\Z_n$}
$(k, n) = 1$

\section*{Задача 2.}

\subsection*{Напишете определението за комутативен пръстен}
Нека $(R, +, \bullet)$ е пръстен. $R$ е комутативен пръстен, ако $\forall a, b \in R: \quad a \bullet b = b \bullet a$.

\subsection*{Напишете определението за пръстен с единица}
Нека $(R, +, \bullet)$ е пръстен. $R$ е пръстен с единица, ако $\exists e \in R: \quad \forall a: \quad a \bullet e = e \bullet a = a$

\subsection*{Напишете определението за област на цялост}
Нека $(R, +, \bullet)$ е комутативен пръстен. Ще казваме, че $R$ е област на цялост, ако $R$ няма ненулеви делители на нулата.

\subsection*{Напишете определението за делител на нулата в пръстен}
Нека $(R, +, \bullet)$ е комутативен пръстен и $a \in R: \quad a \neq 0$. Ще казваме, че $a$ е делител на нулата, ако съществува $b \in R: \quad b \neq 0$, такъв че $a \bullet b = 0$.

\subsection*{Напишете определението за поле}
Поле е комутативен пръстен с $1(\neq 0)$, в който всеки ненулев елемент е обратим.

\subsection*{Напишете определението за тяло}
Тяло е пръстен с единица(различна от нулата), в който всеки ненулев елемент е обратим.

\subsection*{Напишете определението за подпръстен}
Нека $(R, +, \bullet)$ е пръстен и $K$ е непразно подмножество на $R$. Ще казваме, че $K$ е подпръстен на $R$, ако $\forall a, b \in K: \quad a \pm b \in K, a \bullet b \in K$.

\subsection*{Напишете определението за мултипликативната група на пръстен}
Нека $(R, +, \bullet)$ е пръстен с единица. $R^\ast$ е множеството от всички обратими елементи на $R$. $(R^\ast, \bullet)$ е мултипликативна група на пръстена $R$.

\section*{Задача 3.}

\subsection*{Напишете определението за характеристика на поле}
Нека $F$ е поле. Най-малкото $p \in \N$, за което е изпълнено, че $p1_F = 0_F$ се нарича характеристика на полето $F$ и се записва $charF = p$. Ако няма такова число, то $charF = 0$.

\subsection*{Какво число може да бъде характеристиката на едно поле}
Характеристиката на едно поле може да бъде или просто число или 0.

\subsection*{Напишете определението за подполе}
Нека $F$ е поле и $K$ e подмножество на $F$, съдържащо поне два елемента. $K$ е подполе на $F$, ако $\forall a, b \in F: \quad a \pm b \in K, ab \in K \vee a^{-1} \in K$(при $a \neq 0$).

\subsection*{Напишете определението за разширение на поле}
Ако $К$ е подполе на полето $F$, то $K$ съдържа елементите 0 и 1 и също е поле относно операциите в $F$. Ще казваме, че $F$ е разширение на $K$ и ще го бележим с $K \leq F$.

\subsection*{Напишете определението за просто поле}
Нека $F$ е поле. $F$ е просто поле, ако няма собствени(т.е различни от $F$) подполета.

\subsection*{С точност до изоморфизъм, кое поле може да бъде просто подполе на едно поле}
$\Q, \Z_p$\\
Всяко поле с характеристика 0 съдържа единствено подполе, изоморфно на $\Q$.\\
Всяко поле с характеристика $p > 0$ съдържа единствено подполе, изоморфно на $\Z_p$.

\section*{Задача 4.}
\subsection*{Напишете определението за ядро на хомоморфизъм на пръстени}
Нека $R$ и $R'$ са пръстени и $\varphi: R \to R'$ е хомоморфизъм на пръстени. $Ker\varphi = \{a \in R \mid \varphi(a) = 0\}$

\subsection*{Напишете определението за образ на хомоморфизъм на пръстени}
Нека $R$ и $R'$ са пръстени и $\varphi: R \to R'$ е хомоморфизъм на пръстени. $Im\varphi = \{b \in R' \mid \exists a \in R: \quad \varphi(a) = b\}$

\subsection*{Напишете определението за хомоморфизъм на пръстени}
Нека $R$ и $R'$ са пръстени и $\varphi: R \to R'$ е изображение. Ще казваме, че $\varphi$ е хомоморфизъм на пръстени, ако $\forall a, b \in R$:
\begin{itemize}
	\item $\varphi(a + b) = \varphi(a) + \varphi(b)$
	\item $\varphi(ab) = \varphi(a)\varphi(b)$
\end{itemize}

\subsection*{Напишете определението за изоморфизъм на пръстени}
Нека $R$ и $R'$ са пръстени и $\varphi: R \to R'$ е хомоморфизъм на пръстени. Ако $\varphi$ е биекция, то $\varphi$ е изоморфизъм на пръстени. Казваме, че $R$ и $R'$ са изоморфни.

\section*{Задача 5.}

\subsection*{Напишете определението за ляв идеал на пръстен}
Нека $R$ е пръстен и $I$ е непразно подмножество на $R$. Ще казваме, че $I$ е ляв идеал на $R$, ако:
\begin{itemize}
	\item $\forall a, b \in I: \quad a - b \in I$
	\item $\forall a \in I, \forall r \in R: \quad ra \in I$
\end{itemize}

\subsection*{Напишете определението за десен идеал на пръстен}
Нека $R$ е пръстен и $I$ е непразно подмножество на $R$. Ще казваме, че $I$ е десен идеал на $R$, ако:
\begin{itemize}
	\item $\forall a, b \in I: \quad a - b \in I$
	\item $\forall a \in I, \forall r \in R: \quad ar \in I$
\end{itemize}

\subsection*{Напишете определението за двустранен идеал на пръстен}
Нека $R$ е пръстен и $I$ е непразно подмножество на $R$. Ще казваме, че $I$ е двустранен идеал на $R$, ако:
\begin{itemize}
	\item $\forall a, b \in I: \quad a - b \in I$
	\item $\forall a \in I, \forall r \in R: \quad ra \in I$
	\item $\forall a \in I, \forall r \in R: \quad ar \in I$
\end{itemize}

\subsection*{Напишете определението за сума на идеали}
Нека $R$ е пръстен. Ако $I$ и $J$ са идеали на $R$, то множеството $I + J = \{i + j \mid i \in I, j \in J\}$ също е идеал на $R$ и се нарича сума на идеали. Аналогично се дефинира и сума на повече от два идеала.

\subsection*{Напишете определението за главен идеал, породен от елемент, в комутативен пръстен с единица}
Нека $R$ е комутативен пръстен с единица и $a \in R$. Множеството $(a) = \{ar \mid r\in R\}$ се нарича главен идеал на $R$. Лесно се вижда, че $(1) = R$.

\subsection*{Какъв е видът на идеалите в пръстена на целите числа $\Z$}
Всеки идеал е главен, по-точно има вида $n\Z$, където $n$ е естествено число или 0. 

\subsection*{Как се дефинира операцията събиране във факторпръстен}
$\bar{a} + \bar{b} = \bar{a + b}$

\subsection*{Как се дефинира операцията умножение във факторпръстен}
$\bar{a}\bar{b} = \bar{ab}$

\subsection*{Формулирайте теоремата за хомоморфизмите за пръстени}
Нека $R$ и $R'$ са пръстени и $\varphi: R \to R'$ е хомоморфизъм на пръстени и $I = Ker\varphi$. Тогава $I \triq R$ и $R/I \cong Im\varphi$. 

\section*{Задача 6.} 
\subsection*{Докажете, че ако $P$ е поле, то $P$ няма нетривиални идеали (т. е. различни от $\{0\}$ и $P$}
$\{0\} \neq I \triq P$ и $0 \neq a \in I$. Тогава $1 = aa^{-1} \in I$, откъдето $P = (1) \subseteq I$, т.е $I = P$.

\subsection*{Докажете, че ако един комутативен пръстен с единица $P$ няма нетривиални идеали (т.е. различни от $\{0\}$ и $P$), то $P$ е поле}
Нека $0 \neq a \in P$. Тогава $(a) \neq \{0\}$ и значи $(a) = P = (1)$. Следователно съществува елемент $a' \in P" \quad aa' = a'a = 1$, т.е елементът $a$ е обратим и $a^{-1} = a'$. Така всеки ненулев елемент е обратим, откъдето следва, че $P$ е поле.

\subsection*{Докажете, че всяко поле съдържа единствено просто подполе}
Директно се проверява, че сечението на всички подполета на дадено поле е единственото му просто подполе.

\end{document}