\documentclass[10pt]{article}
\usepackage[utf8]{inputenc} 
\usepackage{amsfonts}
\usepackage{amsmath}
\usepackage{amssymb}
\usepackage[utf8]{inputenc}
\usepackage[bulgarian]{babel}
\usepackage{mathtools}

\newcommand*{\Z}{\mathbb{Z}}
\newcommand*{\N}{\mathbb{N}}
\newcommand*{\R}{\mathbb{R}}
\newcommand*{\Q}{\mathbb{Q}}
\newcommand{\triq}{\; \underline{\triangleleft} \;}

\begin{document}

\title{Теория 1}
\author{Валентин Стоянов}
\date{април 2018}
\maketitle

\section*{Задача 1.}

\subsection*{Формулирайте теоремата за делене с частно и остатък за цели числа}
$\forall a, b \in \Z: b \neq 0, \exists q, r \in \Z: a = bq + r, \quad 0 \leq r \leq b$

\subsection*{Напишете определението за най-голям общ делител на две цели числа}
Нека $a, b \in \Z: a \neq 0 \vee b \neq 0$. Най-голям общ делител(НОД) на $a$ и $b$ е числото $d = (a, b)$, ако:
\begin{itemize}
	\item $d \mid a$, $d \mid b$
	\item $d_1 \mid a$, $d_1 \mid b$, то $d_1 \mid d$.
\end{itemize}

\subsection*{Напишете определението за най-малко общо кратно на две цели числа}
Нека $a, b \in \Z: a \neq 0 \vee b \neq 0$. Най-малко общо кратно(НОК) на $a$ и $b$ е числото $d = [a, b]$, ако:
\begin{itemize}
	\item $a \mid d$, $b \mid d$
	\item $a \mid d_1$, $b \mid d_1$, то $d \mid d_1$.
\end{itemize}

\subsection*{Каква е връзката между най-голям общ делител и най-малко общо кратно на две цели числа}
Нека $a, b \in \Z: a \neq 0 \vee b \neq 0$\\
Тогава $[a, b](a, b) = ab$

\subsection*{Напишете равенството на Безу за две цели числа}
Нека $a, b \in \Z: a \neq 0 \vee b \neq 0$\\
Ако $(a, b) = d$, то $\exists u, v \in \Z: au + bv = d$\\
В частност, ако $(a, b) = 1$, то $au + bv = 1$

\subsection*{Формулирайте основната теорема на аритметиката}
Всяко естествено число $n > 1$ се представя по единствен начин(с точност до реда на множителите) като произведение на прости числа.\\
$n = p_1^{\alpha_1} p_2^{\alpha_2} \cdots p_n^{\alpha_n}$

\section*{Задача 2.}

\subsection*{Какво означава едно число да е сравнимо с друго по даден модул}
Нека $n \in \N$, $a, b \in \Z$. Казваме, че $a$ е сравнимо с $b$ по модул $n$ (пише се $a \equiv b \pmod{n}$), ако $n \mid (a - b)$.

\subsection*{Какво представлява класът остатъци $\bar{m} \in \Z_n$}
Това са всички цели числа, които при деление на $n$ дават остатък $m$.

\subsection*{За какви числа $n$ пръстенът $\Z_n$ е поле)}
$\Z_n$ е поле точно когато $n$ е просто число.

\subsection*{За какви числа $k$ класът остатъци $k \in \Z_n$ е обратим елемент на пръстена $\Z_n$}
?????????????????????????????????

\section*{Задача 3.}

\subsection*{Напишете определението за пълна система остатъци по модул $n$}
Всяко множество от $n$ цели числа, които са представители на различни класове(тоест несравними две по две) по модул $n$.

\subsection*{Напишете определението за редуцирана система остатъци по модул $n$}
Всяко множество от $\varphi(n)$ цели числа, които са две по две несравними по модул $n$.

\subsection*{Напишете определението за функция на Ойлер}
Нека $n \in \N$. Функция на Ойлер се бележи с $\varphi(n)$ и представлява броят на естествените числа, които ненадминават $n$ и са взаимно прости с $n$. Тя е мултипликативна, тоест ако $(a, b) = 1$, то $\varphi(ab) = \varphi(a)\varphi(b)$.

\subsection*{Формулирайте теоремата на Ойлер}
Нека $n \in \N$, $z \in \Z$. Тогава $z^{\varphi(n)} \equiv 1 \pmod{n}$.

\subsection*{Формулирайте теоремата на Ферма}
Ако $a \in \Z$, $p$ е просто число и $(a, p) = 1$, то $a^{p - 1} \equiv 1 \pmod{p}$

\section*{Задача 4.}

\subsection*{Напишете определението за това едно число да дели друго}
Ненулево число $b$ дели $a$ (пише се $b \mid a$), ако съществува число $q$, такова че $a = bq$. Ясно е, че остатъкът при делението на $a$ с $b$ е равен на 0.

\section*{Задача 5.}

\subsection*{Формулирайте теоремата на Уилсън}
Ако $p$ е просто число, то $(p - 1)! \equiv -1 \pmod{n}.$

\section*{Задача 6.}

\subsection*{Докажете, че за всяко цяло число $a$ е изпълнено, че $a \mid a$}
$a = aq + r$, където $q = 1, r = 0$, т.е $a = a.1 \Rightarrow a \mid a$

\subsection*{Докажете, че ако $a \mid b$ и $b \neq 0$, то $|a| \leq |b|$}


\subsection*{Докажете, че ако $a \mid b$ и $b \mid c$, то $a \mid c$}
\subsection*{Докажете, че ако $a \mid b$ и $a \mid c$, то $a \mid b + c$}
\subsection*{Докажете, че за всяко цяло число $a$ е изпълнено $a \equiv a \pmod{n}$}
\subsection*{Докажете, че ако $a \equiv b \pmod{n}$, то $b \equiv a \pmod{n}$} 
\subsection*{Докажете, че ако $a \equiv b \pmod{n}$ и $b \equiv c \pmod{n}$, то $a \equiv c \pmod{n}$}
\subsection*{Докажете, че ако $a \equiv b \pmod{n}$ и $c \equiv d \pmod{n}$, то $a \pm c \equiv b \pm d \pmod{n}$}

\section*{Задача 7.}
Нека $M$ е множество и $\circ: M \times M \to M$

\subsection*{Напишете определението за асоциативна операция}
Операцията $\circ$ е асоциативна, ако $\forall a, b, c \in M, \quad a \circ (b \circ c) = (a \circ b) \circ c$

\subsection*{Напишете определението за комутативна операция}
Операцията $\circ$ е комутативна, ако $\forall a, b \in M, \quad a \circ b = b \circ a$

\subsection*{Напишете определението за неутрален елемент}
$\exists e \in M: \quad \forall a \in M, \quad a \circ e = e \circ a = a$

\subsection*{Напишете определението за подгрупа}
Нека $(G, \circ)$ е група и $H \subseteq G, H \neq \emptyset$. $H$ е подгрупа на $G$ ако $H$ е затворено относно операцията в $G$ и обратният за всеки елемент от $H$ е съшо в $H$.

\subsection*{Напишете определението за хомоморфизъм на групи}
Нека $(G_1, \circ_{G_1}))$ и $(G_2, \circ_{G_2})$ са групи и $\varphi: G_1 \to G_2$ е изображение. $\varphi$ е хомоморфизъм на групи, ако за всеки два елемента $a, b \in G_1$ е изпълнено, че $\varphi(a \circ_{G_1} b) = \varphi(a) \circ_{G_2} \varphi(b)$.

\subsection*{Напишете определението за подгрупа породена от подмножество на дадена група}
Нека $(G, \circ)$ е група и $H \subseteq G$. С $\langle H \rangle$ бележим множеството от всички елементи на $G$, които могат да се представят като произведение(сума) на елементи от $H$ или техните обратни(противоположни). $\langle H \rangle \leq G$. Казваме, че $\langle H \rangle$ се поражда от множеството $H$.

\subsection*{Напишете определението за циклична група}
Нека $(G, \circ)$ е група и $g \in G$. Подгрупата $\langle g \rangle = \{g^n \mid n \in \Z\}$, породена от елемента $g$ се състои от всички степени на $g$. $\langle g \rangle$ се нарича циклична група, породена от $g$, а $g$ се нарича неин пораждащ.

\subsection*{Напишете определението за ред на елемент от дадена група}
Нека $(G, \circ)$ е група и $g \in G$. Най-малкото естествено число $r$(ако съществува), за което $g^r = e_G$, се нарича ред на елемента $g$ и се бележи с $r(g)$ или $|g|$. Ако не съществува такова число, то $g$ не е от краен ред и се записва $r(g) = \infty$.

\section*{Задача 8.}

\subsection*{Дайте пример за крайна група}
$S_3$
\subsection*{Дайте пример за безкрайна група}
$\R$

\subsection*{Дайте пример за абелева група}
$\Q$

\subsection*{Дайте пример за неабелева група}
$GL_3(F)$

\subsection*{Дайте пример за крайна циклична група}
$\Z/5\Z$

\subsection*{Дайте пример за безкрайна циклична група}
$\Z$

\section*{Задача 9.}

\subsection*{Напишете определението за съседен клас на група по нейна подгрупа}
Нека $G$ е група, $H \leq G$ и $g \in G$. Множествата $gH = \{gh \mid h \in H\}$ и $Hg = \{hg \mid h \in H\}$ се наричат съответно ляв и десен съседен клас на $G$ по подгрупата $H$. Всеки елемент на $gH(Hg)$ се нарича представител на този съседен клас.

\subsection*{На пишете определението за индекс на подгрупа на дадена група в групата}
Нека $G$ е крайна група и $H \leq G$. Броят на левите(десните) съседни класове на $G$ по $H$, се нарича индекс на $H$ в $G$ и се бележи с $|G:H|$

\subsection*{Формулирайте теоремата на Лагранж}
Нека $G$ е крайна група и $H \leq G$. Тогава $|G| = |H||G:H|$

\subsection*{Напишете определението за нормална подгрупа на дадена група}
Нека $G$ е група и $H \leq G$. Тогава $H$ се нарича нормална подгрупа на $G$(бележи се с $H \triq G$), ако за всеки елемент $g \in G$ е изпълнено, че $gH = Hg$.

\subsection*{Напишете определението за факторгрупа на дадена група по нейна нормална подгрупа}
Нека $G$ е група и $H \triq G$. $G/H$ е множеството от всички леви(десни) съседни класове на $G$ по $H$. $\bullet$ е бинарна операция: $\quad \forall a,b \in G: aH \bullet bH = abH$.

\subsection*{Напишете определението за ядро на хомоморфизъм на групи}
Нека $\varphi: G \to G'$ е хомоморфизъм на групи. Множеството $Ker\varphi = \{g \in G \mid \varphi(g) = e\} \subseteq G$ се нарича ядро на $\varphi$.

\subsection*{Напишете определението за образ на хомоморфизъм на групи}
Нека $\varphi: G \to G'$ е хомоморфизъм на групи. Множеството $Im\varphi = \{a \in G' \mid \exists b \in G: \quad a = \varphi(b)\} \subseteq G'$ се нарича образ на $\varphi$.

\subsection*{Формулирайте теоремата за хомоморфизмите за групи}
Нека $\varphi: G \to G'$ е хомоморфизъм на групи. Тогава $Ker\varphi \triq G$ и $G/Ker\varphi \cong Im\varphi$.

\subsection*{Формулирайте втората теорема за хомоморфизмите за групи}
\subsection*{Формулирайте третата теорема за хомоморфизмите за групи}

\section*{Задача 10.}
Нека $\Omega$ е множество, а $G$ е група.

\subsection*{Напишете определението за действие на група върху множество}
$G$ действа върху $\Omega$, акo:
\begin{itemize}
	\item $ e \in G, \forall x \in \Omega: \quad ex = xe = x$
	\item $\forall g_1, g_2 \in G, \forall x \in \Omega: \quad (g_1g_2)x = g_1(g_2x)$
\end{itemize}

\subsection*{Напишете определението за стабилизатор на елемент от множество при действието на група върху това множество}
Нека $x \in \Omega$. Стабилизатор на $x$ в $G$ е множеството\\ $St_G(x) = \{g \in G \mid gx = x\}$.

\subsection*{Напишете определението за орбита на елемент от множество при действието на група върху това множество}
Нека $x \in \Omega$. Орбита на $x$ е множеството $O(x) = \{gx \mid g \in G\}$.

\subsection*{Напишете как се изразява дължината на орбитата на елемент от множество при действие на група върху това множество чрез редовете на групата и на стабилизатора на елемента}
Нека $x \in \Omega$. Тогава $|O(x)| = |G:St_G(x)| \Rightarrow |O(x)| \mid |G|$.

\subsection*{Напишете определението за клас спрегнати елементи на елемент от дадена група}
Нека $a, b \in G$. $a$ и $b$ се наричат спрегнати, ако съществува $g \in G: \quad b = g^{-1}ag$.

\subsection*{Напишете определението за централизатор на елемент от дадена група}
Централизатор на елемента $a \in G$ e множеството от елементи на $G$, които комутират с $a$. $C_G(a) = \{g \in G \mid ag = ga\} \leq G$

\subsection*{Напишете определението за център на група}
Множеството $Z(G) = \{a \in G \mid \forall g \in G: ga = ag\}$ се нарича център на групата $G$. $Z(G) = G \Leftrightarrow G$ е абелева група.

\subsection*{Напишете формулата за класовете}


\subsection*{Формулирайте теоремата на Кейли}
Всяка крайна група от ред $n$ е изоморфна на подгрупа на симетричната група $S_n$.

\end{document}