\documentclass[12pt]{article}
\usepackage[utf8]{inputenc} 
\usepackage{amsmath}
\usepackage[utf8]{inputenc}
\usepackage[bulgarian]{babel}

\begin{document}

\title{Решени задачи от въведение в теория на числата}
\author{Валентин Стоянов}
\date{март 2018}
\maketitle

\section*{Задача 1.}
Да се докаже, че за всяко естествено число $n$, числото $n^3 + 11n$ се дели на 6.\\
Доказателство по индукция:\\
\subsection*{1. База $n = 1$}
$6 \mid 12$ да.
\subsection*{2. Индукционна хипотеза}
Допускаме, че $6 \mid n^3 + 11n$
\subsection*{3. Индукционна стъпка}
Проверяваме дали твърдението е вярно за $n+1 \Rightarrow$\\
$\Rightarrow 6 \mid (n+1)^3 + 11(n+1)$\\
$\Leftrightarrow 6 \mid n^3 +3n^2 + 3n + 1 + 11n + 11$\\
$\Leftrightarrow 6 \mid 3n^2 + 3n$\\
$\Leftrightarrow 6 \mid 3n(n + 1)$\\
$\Leftrightarrow 2 \mid n(n + 1)$\\
Но $n(n + 1)$ са две поредни числа $\Rightarrow$ винаги поне едно е четно
$\Rightarrow 2 \mid n(n + 1)$

\section*{Задача 2.}
Да се докаже, че за всяко естествено число $n$, числото $3^{2^n} - 1$ се дели на $2^{n + 2}$.\\
Доказателство по индукция:\\
\subsection*{1. База $n = 1$}
$8 \mid 8$ да.
\subsection*{2. Индукционна хипотеза}
Допускаме, че $2^{n + 2} \mid 3^{2^n} - 1$
\subsection*{3. Индукционна стъпка}
Проверяваме дали твърдението е изпълнено за $n+1 \Rightarrow$\\
$2^{(n + 1) + 2} \mid 3^{2^{n + 1}} - 1$\\
$\Leftrightarrow 2^{(n + 2) + 1} \mid 3^{2^{n + 1}} - 1$\\
$\Leftrightarrow 2^{n + 2}2^1 \mid 3^{2^n2^1} - 1$\\
$\Leftrightarrow 2^{n + 2}2 \mid 3^{2^n2} - 1$\\
$\Leftrightarrow 2^{n + 2}2 \mid (3^{2^n})^2 - 1$\\
$\Leftrightarrow 2^{n + 2}2 \mid (3^{2^n} - 1)(3^{2^n} + 1)$\\
$\Leftrightarrow 2 \mid (3^{2^n} + 1)$\\
$3^{2^n}$ е нечетно число $\Rightarrow 3^{2^n} + 1$ e четно $\Rightarrow$\\
$2 \mid (3^{2^n} + 1)$

\end{document}
