\documentclass[12pt]{article}
\usepackage[utf8]{inputenc} 
\usepackage{amsmath}
\usepackage[utf8]{inputenc}
\usepackage[bulgarian]{babel}

\begin{document}

\title{Решени задачи от въведение в теория на числата}
\author{Валентин Стоянов}
\date{март 2018}
\maketitle

\section*{Задача 1.}
Да се докаже, че за всяко естествено число $n$, числото $n^3 + 11n$ се дели на 6.\\
Доказателство по индукция:\\
\subsection*{1. База $n = 1$}
$6 \mid 12$ да.
\subsection*{2. Индукционна хипотеза}
Допускаме, че $6 \mid n^3 + 11n$
\subsection*{3. Индукционна стъпка}
Проверяваме дали твърдението е вярно за $n+1 \Rightarrow$\\
$\Rightarrow 6 \mid (n+1)^3 + 11(n+1)$\\
$\Leftrightarrow 6 \mid n^3 +3n^2 + 3n + 1 + 11n + 11$\\
$\Leftrightarrow 6 \mid 3n^2 + 3n$\\
$\Leftrightarrow 6 \mid 3n(n + 1)$\\
$\Leftrightarrow 2 \mid n(n + 1)$\\
Но $n(n + 1)$ са две поредни числа $\Rightarrow$ винаги поне едно е четно
$\Rightarrow 2 \mid n(n + 1)$

\section*{Задача 2.}
Да се докаже, че за всяко естествено число $n$, числото $3^{2^n} - 1$ се дели на $2^{n + 2}$.\\
Доказателство по индукция:\\
\subsection*{1. База $n = 1$}
$8 \mid 8$ да.
\subsection*{2. Индукционна хипотеза}
Допускаме, че $2^{n + 2} \mid 3^{2^n} - 1$
\subsection*{3. Индукционна стъпка}
Проверяваме дали твърдението е изпълнено за $n+1 \Rightarrow$\\
$2^{(n + 1) + 2} \mid 3^{2^{n + 1}} - 1$\\
$\Leftrightarrow 2^{(n + 2) + 1} \mid 3^{2^{n + 1}} - 1$\\
$\Leftrightarrow 2^{n + 2}2^1 \mid 3^{2^n2^1} - 1$\\
$\Leftrightarrow 2^{n + 2}2 \mid 3^{2^n2} - 1$\\
$\Leftrightarrow 2^{n + 2}2 \mid (3^{2^n})^2 - 1$\\
$\Leftrightarrow 2^{n + 2}2 \mid (3^{2^n} - 1)(3^{2^n} + 1)$\\
$\Leftrightarrow 2 \mid (3^{2^n} + 1)$\\
$3^{2^n}$ е нечетно число $\Rightarrow 3^{2^n} + 1$ e четно $\Rightarrow$\\
$2 \mid (3^{2^n} + 1)$

\section*{Задача 3.}
Да се докаже, че за всяко естествено число $n$\\
$55^2 \mid 81^{n+1} + (55n -81)136^n$\\
Доказателство по индукция:\\
\subsection*{1. База $n = 0$}
$55^2 \mid 0$ да.
\subsection*{2. Индукционна хипотеза}
Допускаме, че $55^2 \mid 81^{n+1} + (55n -81)136^n$
\subsection*{3. Индукционна стъпка}
Проверяваме дали твърдението е изпълнено за $n+1 \Rightarrow$\\
$55^2 \mid 81^{n+2} + (55(n + 1) -81)136^{n+1}$\\
$\Leftrightarrow 55^2 \mid 81^{n + 1}81 + (55n + 55 - 81)136^n136$\\
$\Leftrightarrow 55^2 \mid 81^{n + 1}81 + (55n + 55 - 81)(81 + 55)136^n$\\
$\Leftrightarrow 55^2 \mid 81^{n + 1}81 + (55n \times 81 + 55^2n + 55 \times 81 + 55^2 - 81^2 - 81 \times 55)136^n$\\
$\Leftrightarrow 55^2 \mid 81^{n + 1}81 + (55n \times 81 + 55^2n + 55^2 - 81^2)136^n$\\
$\Leftrightarrow 55^2 \mid 81^{n + 1}81 + ((55n - 81)81  + 55^2n + 55^2)136^n$\\
$\Leftrightarrow 55^2 \mid 81^{n + 1}81 + ((55n - 81)81  + 55^2(n + 1))136^n$\\
$\Leftrightarrow 55^2 \mid 81^{n + 1}81 + (55n - 81)136^n81  + 55^2(n + 1)136^n$\\
От индукционната хипотеза и свойствата за делимост $\Rightarrow$\\
$\Rightarrow 55^2 \mid 81^{n+2} + (55(n + 1) -81)136^{n+1}$

\section*{Задача 4.}
Да се намери d = НОД(a, b) и цели числа u, v, за които, au + bv = d, ако:\\
\subsection*{a) a = 315, b = 72;}
Решение:\\
Намираме НОД чрез алгоритъма на Евклид $\Rightarrow$\\
$\Rightarrow$ d = НОД(a, b) = НОД(315, 72)\\
315 : 72 = 4(остатък 27)\\
72 : 27 = 2(остатък 18)\\
27 : 18 = 1(остатък 9)\\
18 : 9 = 2(остатък 0)\\
НОД е последният ненулев остатък $\Rightarrow d = НОД(315, 72) = 9$\\
От Безу знаем, че ако НОД(a, b) = d, то съществуват цели числа u и v, такива, че au + bv = d.\\

\subsection*{б) a = 975, b = 308;}
Решение:\\
Намираме НОД чрез алгоритъма на Евклид $\Rightarrow$\\
$\Rightarrow$ d = НОД(a, b) = НОД(975, 308)\\
975 : 308 = 3(остатък 51)\\
308 : 51 = 6(остатък 2)\\
51 : 2 = 25(остатък 1)\\
25 : 1 = 25(остатък 0)\\
НОД е последният ненулев остатък $\Rightarrow d = НОД(975, 308) = 1$\\
От Безу знаем, че ако НОД(a, b) = d, то съществуват цели числа u и v, такива, че au + bv = d.\\



\end{document}
